\documentclass{article}
\usepackage{geometry}
\usepackage[usenames,dvipsnames,svgnames]{xcolor}
\usepackage{graphicx} % Required for inserting images
\usepackage{amsmath}
\usepackage{biblatex}

\addbibresource{shortest_path_problems.bib}

\begin{document}

\thispagestyle{empty}

\vspace*{+2em}
\begin{center}
    {\LARGE SUMMARY OF DREYFUS' \\
    AN APPRAISAL OF SOME SHORTEST-PATH \\
    ALGORITHMS\\}
\vspace*{+4em}


\vspace*{+3em}
\begin{figure}[h]
    \centering
    \includegraphics[width=0.25\textwidth]{graphics/UU_logo_color.png}
\end{figure}

\vspace*{+3em}
Graph Theory \\
Uppsala University Ångströms Laboratory\\
Department of Math\\
\vspace*{+2em}
William Paradell, Mira Åkerman, Max Callenmark\\

\vspace*{+10em}
12\--12\--2023

\end{center}

\newpage

\section*{Introduction}

This paper is a short summary of Stuart E. Dreyfus' paper 
\textit{An appraisal of some shortest-path algorithms}\cite{Dreyfus_1969}, where five 
different shortest paths problems are discussed and how good 
different algorithms are at solving the problems.  \\
\indent We strive with this summary to summarize the arguments presented
pertaining to the different problems, as well as try to give an 
overview of the algorithms presented to simplify understanding. 

\section*{Part 1: The shortest-path problem}
To start off the shortest path problem Dreyfus presents the problem in greater detail
as well as a computationally efficient algorithm solving the problem by Whiting and Hillier.
The algorithm will be skipped since it's presented in relatively great detail in Dreyfus's paper;
but it's worth mentioning the computational complexity of it, since it's referenced many times
while he discusses other algorithms. The algorithm requires $\frac{N(N-1)}{2}$ additions and $N(N-1)$ 
comparisons to solve the problem; where N is the number of nodes. Next he presents other
algorithms pertaining to the problem and starts off with Pollack and Wiebenson's paper\cite{pollack_wibenson}, but does not present the algorithm, which will be done in this summary, to 
further simplify understanding of the points presented pertaining to it. In the paper a duality 
is presented with Minty's and the Ford-Fulkerson algorithm (where the shortest path is a special case of
the Ford-Fulkerson algorithm for network flows), since the Ford-Fulkerson algorithm is relatively well
known, only Minty's algorithm will be presented before the duality is explained. Minty's algorithm 
as it is presented in Pollack and Wiebenson's paper: 
\vspace{2mm}

\noindent For the shortest-path problem between two specified cities (nodes).
Label the starting city `A' and the desired destination city `B' (the terminating city). \\
\indent \textbf{1.} Label the starting city A with the distance `zero' \\
\indent \textbf{2.} Look at all one-way streets with their `tails' labeled
and their `heads' unlabeled. For each such street, form the sum of the label and the length. Select a
street making this sum a minimum and place a check-mark beside it. Label the
`head' city of this street with the sum. Return to the beginning of 2 

\vspace{2mm}

\noindent According to Dreyfus, Minty's algorithm is combined with the Ford-Fulkerson
algorithm to make the \textit{Minty-Ford-Fulkerson} algorithm which is stated to be
easily programmed, but on the other hand significantly slower than the algorithm he
started off by presenting, needing a total of $\frac{N^{3}}{6}$ additions and comparisons, versus
the $\frac{N(N-1)}{2}$ additions and $N(N-1)$ comparisons of the previous algorithm. Dreyfus does
present a paper by Whiting and Hillier that discusses a modification to the algorithm
that speeds up the Minty-Ford-Fulkerson algorithm, but according to Dreyfus, it's not
enough in order to compete with Dijkstra's algorithm.

\newpage

\section*{Part 3: Determination of the second-shortest path}
Dreyfus also talks about the problem of finding the second-shortest 
path through a network. To solve this problem he talks about different methods. 
Firstly Dreyfus refers to the method described by Hoffman and Pavley\cite{hoffman_pavley}. 
The method says that the second-shortest path between specified  initial and terminal 
nodes is the deviation from the shortest path. The deviation from the shortest path is 
defined “to be a path that coincides with the shortest path ($j$ may be the origin or the
terminal node), then deviates directly to some node $k$ not the next node of the shortest
path, and finally proceeds from $k$ to the fixed terminal node via the shortest path 
from $k$”. Using this method will approximately use $MK$ additions and comparisons, beyond
the ones needed to find the shortest path, where $M$ is the average node outgoing 
edges and $K$ is the number of edges in the shortest path. 
\vspace{2mm}
Secondly Dreyfus refers to the method described by Bellman and Kalaba
\cite{bellman_kalaba}. This method uses the equation:

\begin{equation}
    \begin{aligned}
        & v_{i} = \min \left[
        \begin{aligned} 
            & \min_{i \neq j}\!_2(d_{ij} + u_{j}), \\
            & \min_{j \neq i}\!_1(d_{ij} + v_{j})
        \end{aligned} 
        \right],\\
        & v_{N} = \min_{i}\!_1[d_{Ni} + u_{i}].
    \end{aligned}
    \hspace*{3em} (i = 1, \dots, N - 1) 
\end{equation}

\noindent In this equation $u_i$ is defined as the length of the shortest path from the node $i$
to the terminal node $N$ and $v_i$ as the length of the second shortest path. 
The term $\min_k(x_1, \dots, x_n)$ is defined as the $k$:th smallest value of the quantities $x_i$.
The term $\min \!_2(d_{ij}+u_j)$ determines the best path beginning in node $i$ 
and deviating from the shortest path in node $i$ and the term $\min\!_1(d_{ij}+v_j) $
evaluates the best path with any first edge and the second best continuation. 
Bellman and Kalaba recommends solving the equation above using an iterative procedure 
until convergence by elevating $v_i$ on the left with $(k+1)$ and $v_j$ on the right with $(k)$. 
Using this method will need an average of $NML$ additions and comparisons, where $L$ is the 
number of iterations needed. Dreyfus compares this method to Hoffman and Pavley and 
says that Hoffman and Pavley's method is the better one because it requires less 
calculations.
\vspace{2mm}
Dreyfus also writes about a method from Pollack's\cite{m_pollack} paper to find the 
third-shortest path from all initial nodes to N. We get this path by following

\begin{equation}
    w_i = \min \left[
    \begin{aligned}
        & d_{ik} + v_k \\
        & d_{im} + v_m \\
        & \min_{j \neq i} \!_3 [d_{ij} + u_j]
    \end{aligned}
    \right].
\end{equation}

\noindent Let $w_i$ represent the length of the third-shortest path from $i$,
$v_i$ the second-shortest path and $u_i$ the shortest path. Let $k$ be the verticy 
following $i$ on the shortest path and $m$ the vertices following $i$ on the second shortest path.
If $u_i$ and $v_i$ is determined then $w_i$ can be computed node by node, this 
by first computing nodes that are one egde away from $N$, then two etc. 

\newpage
\section*{Part 4: Time-dependent length of arcs}
Dreyfus refers to a paper by Cooke and Halsey\cite{halsey} that studied the problem of finding 
the fastest paths between cities, where travel time depends on departure 
from the starting city. In the paper they refer to two cities
as $i$ and $j$ and denote $d_{ij}(t)$ as the travel time where $t$ is the time of 
departure and defines a function $f_i(t)$ as the minimum travel time to $N$. 
\begin{equation}
    \begin{aligned}
        & f_i(t) = \min_{j \neq i}[d_{ij}(t)+f_j(t+d_{ij}(t))] \\
        & f_N(t) = 0
    \end{aligned}
\end{equation}
The authors present an iterative algorithm for finding the quickest 
paths from all cities to city $N$. Assume all cities are connected, even if it 
takes an infinite time. The algorithm can be summarized as follows:
\begin{enumerate}
    \item Start at city i at time 0 
    \item Define T as the maximum time taken over all initial cities 
    \item Use tentative and permanent labels for nodes (cities) 
    \item Permanently label the initial node $i$ as 0 and all the others nodes as infinity
    \item Tentatively label nodes with the minimum of the current label and the sum of the current label and the time taken to travel to that node.
    \item Find the minimum, non-permanent node label and declare it permanent.
    \item Use the selected node to update labels at other tentatively labeled cities.
    \item Repeat steps 6\--7 until city $N$ is permanently labeled.
\end{enumerate}

\noindent The procedure requires most $N^{2}T^{2}$additions and comparisons.
After most $N^{2}$ comparisons and $\frac{N^{2}}{2}$ additions, 
the quickest paths to all nodes, including $N$, are determined. A similar 
result is achieved by using a variant of the Dijkstra algorithm, where $\frac{N^{2}}{2}$
additions and $2N^{2}$comparisons are required. If the quickest path from all 
cities to $N$ is wanted the algorithm must be repeated $N-1$ times. This 
procedure compares favorably in computation and required assumptions with 
Cooke and Halsey's algorithm.

\newpage

\section*{Part 5: Shortest path visiting specified nodes}
The last problem Dreyfus discusses in the paper is the problem of
finding the shortest path between two specified nodes 1 and $N$ that pass
through $k-1$ nodes $(2,3,\dots,k \leq N-1)$. Dreyfus critiques a solution given to 
the problem by Saksena and Kumar\cite{kumar}. The solution proposed by Saksena and 
Kumar assumes that the shortest path from a specified node $i$ to the destination 
node $N$, passing through at least $p$ specified nodes can be broken down into two problems:
\begin{enumerate}
    \item Finding the shortest path from $i$ to some specified node $j$.
    \item Finding the shortest path from $j$ to $N$, passing through at 
    least $p-r-1$ specified nodes, where $r$ is the number of 
    specified nodes on the shortest unrestricted path from $i$ to $j$.
\end{enumerate}

\noindent Dreyfus claims that this procedure is flawed. If a specified 
node exists on the shortest path from $i$ to $j$ it also exists on 
the subsequent path from $j$ to $N$ and is therefore counted twice. 
If any potential path associated with node $j$ does not allow node 
duplication the scenario is not allowed, and the option of initially 
traveling from $i$ to $j$ via the shortest path is eliminated from 
consideration. Dreyfus further claims that the procedure overlooks 
the possibility that a slightly longer continuation from $j$, passing 
through at least $p-r-1$ specified nodes and avoiding node duplication, 
could result in a better path than the best remaining path that satisfies 
the conditions outlined above. Dreyfus asserts that the problem can be 
correctly solved as follows assuming paths with loops are admissible: 

\begin{enumerate}
    \item Solve the shortest path problem for the $N$-node network for all pairs of
    initial and final nodes. Let $d_{ij}$ represent the length of the shortest path 
    from node $i$ to $j$.
    \item Solve the $k+1$ “traveling-salesman” problem for the shortest path from 
    1 to $N$ passing through nodes $2, 3, \dots,k$ where the distance from node 
    $i$ to $j$ is $d_{ij}$. 
\end{enumerate}


\section*{Notes}
Some notes that didn't fit with the flow of the summary.
In part 1 it's worth mentioning that it's not clear what algorithm from
Pollack and Wiebenson's paper Dreyfus has named the Minty-Ford-Fulkerson
algorithm, as it's not explicitly stated in the paper that there's a combination
of Minty's algorithm and the Ford-Fulkerson algorithm; but in their section \textit{Dual
Solutions} they do discuss a changed Ford-Fulkerson algorithm. Dreyfus's
paper originally covers 5 parts but because of group problems we decided to
only cover 4 out of the 5 parts. The part we skipped was part 2, 
\textit{The shortest paths between all pairs of nodes of a network}. 

\newpage
\printbibliography[]

\end{document}

 