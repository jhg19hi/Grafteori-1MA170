\documentclass[nobib]{tufte-handout}

\title{Lecture 16: Edge-colourings and Ramsey theory $\cdot$ 1MA170}

\author[Vilhelm Agdur]{Vilhelm Agdur\thanks{\href{mailto:vilhelm.agdur@math.uu.se}{\nolinkurl{vilhelm.agdur@math.uu.se}}}}

\date{6 December 2023}


%\geometry{showframe} % display margins for debugging page layout

\usepackage{graphicx} % allow embedded images
  \setkeys{Gin}{width=\linewidth,totalheight=\textheight,keepaspectratio}
  \graphicspath{{graphics/}} % set of paths to search for images
\usepackage{amsmath}  % extended mathematics
\usepackage{booktabs} % book-quality tables
\usepackage{units}    % non-stacked fractions and better unit spacing
\usepackage{multicol} % multiple column layout facilities
\usepackage{lipsum}   % filler text
\usepackage{fancyvrb} % extended verbatim environments
  \fvset{fontsize=\normalsize}% default font size for fancy-verbatim environments

\usepackage{color,soul} % Highlights for text

% Standardize command font styles and environments
\newcommand{\doccmd}[1]{\texttt{\textbackslash#1}}% command name -- adds backslash automatically
\newcommand{\docopt}[1]{\ensuremath{\langle}\textrm{\textit{#1}}\ensuremath{\rangle}}% optional command argument
\newcommand{\docarg}[1]{\textrm{\textit{#1}}}% (required) command argument
\newcommand{\docenv}[1]{\textsf{#1}}% environment name
\newcommand{\docpkg}[1]{\texttt{#1}}% package name
\newcommand{\doccls}[1]{\texttt{#1}}% document class name
\newcommand{\docclsopt}[1]{\texttt{#1}}% document class option name
\newenvironment{docspec}{\begin{quote}\noindent}{\end{quote}}% command specification environment

\include{mathcommands.extratex}

\begin{document}

\maketitle% this prints the handout title, author, and date

\begin{abstract}
\noindent
Continuing our previous work on vertex colourings, we now consider the related notion of edge colourings. Then we discuss the basic notions of Ramsey theory.
\end{abstract}

\section{Edge-colourings}

We already saw the definition of an edge-colouring in the exercise session, but let us restate it here as well:

\begin{definition}
    Let $G = (V,E)$ be a graph. A \emph{proper}\sidenote[][]{Unlike for vertex colourings, we will actually be interested in improper edge colourings more often than proper ones, so we choose the opposite convention of including the word proper and omitting the word improper for them.} \emph{$k$-edge-colouring} is a function $c: E \to [k]$ such that no two edges which are incident to each other (i.e. share an endpoint) are assigned the same colour. If we do not have this restriction on incident edges, we call it just an (improper) edge colouring.

    The \emph{edge-chromatic number} of $G$, denoted $\chi_1(G)$,\sidenote[][]{This is sometimes also called the \emph{chromatic index} of $G$. The $1$ in the notation indicates that edges are one-dimensional -- if we ever need to refer to both the chromatic number and the edge-chromatic number at the same time, we may thus denote the chromatic number by $\chi_0(G)$, since vertices are zero-dimensional. In some texts the edge-chromatic number is denoted by $\chi'(G)$, but $\chi$ and $\chi'$ look way too similar in \LaTeX\ and on a blackboard, so let us avoid that notation.} is the smallest integer $k$ such that $G$ has a proper $k$-edge-colouring.
\end{definition}

\begin{remark}
    We notice immediately that for a proper edge-colouring, the colour classes are all matchings, just like how for a vertex colouring the colour classes are all independent sets. In fact, of course, a proper edge-colouring is a vertex colouring of the line graph, so this is just an instance of the correspondence between matchings on $G$ and independent sets in $L(G)$.
\end{remark}

We can also see that if $v$ is a vertex of maximum degree, its incident edges must all have different colours, so we must have that $\chi_1(G) \geq \Delta$.\sidenote[][]{We could also phrase this as that these $\Delta$ edges form a clique in the line graph, so the chromatic number of the line graph must be at least $\Delta$, and $\chi(L(G)) = \chi_1(G)$.} In fact this bound is attained for bipartite graphs.

\begin{theorem}[König's line-colouring theorem, 1916]
    Every bipartite graph $G$ with maximum degree $\Delta$ has edge-chromatic number $\chi_1(G) = \Delta$.

    \begin{proof}
        We prove the result by induction on the number $m = \abs{E}$ of edges. The base case of $m = 0$ is trivial.

        So, let $G = (V,E)$ be bipartite, with $m$ edges and maximum degree $\Delta$. Pick an arbitrary edge $v \sim w$, and let $G'$ be the graph obtained from $G$ by deleting this edge.

        Now $G'$ is a bipartite graph on fewer edges, and so by the induction hypothesis it can be properly edge-coloured with $\Delta(G') \leq \Delta$ colours, so we pick such a colouring.

        Each of the vertices $v$ and $w$ are incident to at most $\Delta-1$ edges in $G'$, so there must exist a colour $i$ not used by any edge incident to $v$, and likewise a colour $j$ not incident to $w$. If $i = j$ we can just colour the edge $v \sim w$ in this colour, so assume that $i \neq j$.

        The rest of the argument essentially proceeds by considering the Kempe chains of the colouring, and doing a Kempe change. Of course, we defined Kempe chains only for \emph{vertex} colourings, so we need to consider edge-induced subgraphs instead of induced subgraphs.

        So consider the graph $G\langle i,j\rangle$ -- the edge-induced subgraph of $G$ consisting only of the edges coloured with $i$ or with $j$. If we could show that $v$ and $w$ are in different connected components of this graph, then we could swap the two colours $i$ and $j$ in the component containing $w$, enabling us to colour the edge $v \sim w$ in colour $i$.

        So suppose for contradiction that there were a path $P$ from $v$ to $w$ in $G\langle i,j\rangle$. Then, since $v$ is not incident to any edge of colour $i$ by assumption, the path $P$ has to start with an edge coloured $j$. Likewise, $w$ is not incident to any edge coloured $j$, so $P$ has to end with an edge coloured $i$.

        So if we just look at the sequence of colours of edges in $P$, it has to look like
        $$j\,i\,j\,i\ldots j\,i\,j\,i$$
        and so in particular we see that it must be of even length, since all the $i$s occur in even-numbered positions and the sequence ends on an $i$.

        Therefore, the path $P$ together with the edge $v \sim w$ forms a cycle in $G$ of odd length. This, however, is impossible, since a graph is bipartite if and only if it contains no cycles of odd length.\sidenote[][]{This, unfortunately, is not a statement we have had occasion to prove in the course. Fortunately, it makes for a very nice exercise.
        
        \begin{xca}Prove this characterization of bipartite graphs.\end{xca}}
    \end{proof}
\end{theorem}

So we have seen that the obvious bound is sharp for the bipartite graphs. What other values can the edge-chromatic number take? It turns out that the range of values is far more restricted than for vertex colourings.

\begin{theorem}[Vizing, 1964]
  For any graph $G$ of maximum degree $\Delta$, it holds that $\chi_1(G) \in \{\Delta,\, \Delta+1\}$. A graph such that $\chi_1(G) = \Delta$ is said to be of \emph{class one}, and a graph such that $\chi_1(G) = \Delta + 1$ is said to be of \emph{class two}.\sidenote[][]{It turns out that there is no known classification of which graphs are of which class, but some partial results are known.\begin{xca}For which values of $n$ is $K_n$ of class one and for which of class two?\end{xca}}
\end{theorem}

\section{Ramsey theory}

Let us now dip our toes into the field of Ramsey theory. It is a large research area with many questions to be posed, and we will give it a shamefully short treatment. Let us start by stating the second most elementary definition in the area.\sidenote[][]{In the exercises, we saw \emph{the} most basic definition. Let's make it mildly more interesting here.}

\begin{definition}
  For any integers $r$ and $k$, the \emph{Ramsey number} $R(r,k)$ is the smallest integer $n$ such that every edge-colouring of $K_n$ using only the two colours red and blue contains a red $K_r$ or a blue $K_k$.\sidenote[][]{By ``containing a monochromatic $K_r$'' we mean that for some colour $i$, the edge-induced subgraph $K_n\langle c^{-1}(i)\rangle$ contains an $r$-clique.} Equivalently, it is the smallest integer $n$ such that any graph on $n$ vertices contains either an independent set of size $k$ or has a clique of size $r$.
\end{definition}

\begin{remark}
  Why are the two things we stated equivalent? Given any graph $G = (V,E)$ on $n$ vertices, we can edge-colour the complete graph $K_n$ by colouring the edge $i \sim j$ red if $i \sim j \in E$, and blue otherwise. Then a red clique in the colouring is a clique in $G$, and a blue clique is an independent set in $G$.
\end{remark}

Of course, these definitions contain the hidden assumption that these numbers actually exist. A priori there might be some $r$ and $k$ such that there exist arbitrarily large graphs $G$ with $\alpha(G) < k$ and $\omega(G) < r$. So let us prove that these numbers are indeed finite.

\begin{lemma}\label{lemma:ramsey_recursion_ub}
  For all $m$ and $n$, it holds that
  $$R(m,n) \leq R(m-1,n) + R(m, n-1).$$

  \begin{proof}
    To keep the notation manageable, let us let $s = R(m-1, n)$ and $t = R(m, n-1)$. Consider some edge-colouring of $K_{s+t}$, and pick an arbitrary vertex $v$. By the pigeonhole principle, this $v$ has either $s$ incident red edges or $t$ incident blue edges.\sidenote[][]{If it did not, the total number of vertices in the $K_{s+t}$ would be upper bounded by
    $$1 + (s - 1) + (t - 1) = s + t - 1$$
    which is a contradiction.}

    So assume without loss of generality that it has $s$ neighbours by red edges,\sidenote[][]{The argument if it instead had $t$ neighbours by blue edges is entirely the same.} and let $S$ be this set of neighbours. Then the induced subgraph of just these neighbours is a $K_s$, which thus must contain either a blue $K_n$ or a red $K_{m-1}$. In the first case, we are done, and in the second case, the red $K_{m-1}$ together with $v$ forms a red $K_m$, and we are again done.
  \end{proof}
\end{lemma}

\begin{corollary}
  For all $m$ and $n$, $R(m,n)$ is finite, and in particular,
  $$R(m,n) \leq \binom{m + n - 2}{m - 1}.$$

  \begin{proof}
    It is easy to see that $R(m,1) = R(1,m) = 1$, so the finiteness is immediate by induction. To see the more explicit bound, we observe that it is nearly as immediate that $R(m,2) = m$ and $R(2,n) = n$, and then do a standard proof by induction, which we omit.\sidenote[][]{It is just a version of the very basic ``prove this formula with binomial coefficients'' proofs that one sees when first introduced to induction, there are no fancy ideas here. Do work out the details if you like.}
  \end{proof}
\end{corollary}

Having seen the upper bound, let us now also establish a lower bound on this quantity. To see a lower bound, it suffices to find a single edge-colouring of a $K_n$ which contains no monochromatic $K_k$. This time, we will get to use the probabilistic method -- the idea is just to consider a random colouring, and show that the probability that it has a monochromatic clique is less than one.\sidenote[][]{This is one of the early proofs by Erd\H{o}s using the probabilistic method, that really made its power clear.}

\begin{theorem}[Erd\H{o}s, 1947]
  If $\binom{n}{k}2^{1 - \binom{k}{2}} < 1$, then $R(k,k) > n$. In particular, $R(k,k) \geq \floor{2^{k/2}}$ for all $k \geq 3$.\sidenote[][]{We only prove the first statement -- the second follows just by seeing that
  $$\binom{\floor{2^{k/2}}}{k}2^{1 - \binom{k}{2}} < 1,$$
  which is not a calculation we want to have to do.}

  \begin{proof}
    As advertised, we consider a random edge-colouring of a $K_n$ -- we just assign each edge red or blue with equal probability.

    So, for each set $S \in \binom{[n]}{k}$, let $R_S$ be the event that $K_n[S]$ is monochrome red, and $B_S$ the event that it is monochrome blue. Clearly $\Prob{R_S} = \Prob{B_S} = 2^{-\binom{k}{2}}$.
    
    Then the event that the colouring has a monochrome $K_k$ is just
    $$\bigcup_{S \in \binom{[n]}{k}} R_S \cup B_S,$$
    and so we can use a union bound to see that
    \begin{align*}
      \Prob{\bigcup_{S \in \binom{[n]}{k}} R_S \cup B_S} &\leq \sum_{S \in \binom{[n]}{k}} \Prob{R_S} + \Prob{B_S} = \binom{n}{k}2^{1 - \binom{k}{2}}
    \end{align*}
    which by assumption is less than one, and so there must exist an outcome in which the event does not occur, which will be our example.
  \end{proof}
\end{theorem}

\section{Exercises}


%\bibliography{references}
%\bibliographystyle{plainnat}

\end{document}
